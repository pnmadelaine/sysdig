L'horloge est divisée en plusieurs parties.

\begin{enumerate}
\item Tout d'abord, on charge le timestamp et on convertit les années pour commencer à compter le temps (approximativement) au bon moment.

Cette fonction est realisée par le code suivant (le timestamp est chargé en RAM à l'adresse 28):
\begin{verbatim}
  lw $at,$zero,28
  li $t0,0
  li $t1,0
  li $t2,0
  li $t3,1
  li $t4,3
  li $t5,1
  li $t6,1970
  
  lui $ra, 1926
  addiu $ra,$ra,8064
  divu $at,$ra
  mflo $a1
  sll $ra,$a1,2
  addu $t6,$t6,$ra
  
  mfhi $s0
  lui $ra, 481
  addiu $ra,$ra,13184
  divu $s0,$ra
  mflo $a2
  addu $t6,$t6,$a2

  mfhi $ra
  subu $fp,$at,$ra

  
  li $ra,5
  multu $ra,$a1
  mflo $ra
  addu $t4,$t4,$ra  
  addu $t4,$t4,$a2
  
  li $ra,7
  divu $t4,$ra
  mfhi $t4
  
  j init
\end{verbatim}

où les instructions
\begin{verbatim}
  mfhi $ra
  subu $fp,$at,$ra
\end{verbatim}
chargent dans fp le nombre de secondes précalculées

\item
Puis, on effectue le calcul du temps à proprement parler dans le code suivant :
\begin{verbatim}
second:
  lw $at,$zero,28
  beq $at,$fp,second

  addiu $fp,$fp,1

  addiu $t0,$t0,1
  beq $t0,$a0,minute
  sw $t0,$zero,0
  j second
\end{verbatim}

On fait appel à minute lorsque 60 seconde sont atteintes (\$a0 contient 60), puis à heure lorsque 60 minutes sont atteintes, et ainsi de suite.

Les instructions 
\begin{verbatim}
lw $at,$zero,28
beq $at,$fp,second
\end{verbatim}

empêchent le processus d'incrémenter les secondes tant que le timestamp ne change pas (at contient le nouveau timestamp, et fp le timestamp calculé). Elles permettent cependant de rattraper l'heure actuelle dans le processus en temps réel.

L'horloge rapide consiste à enlever ces 2 instructions.

\item
Enfin, les données importantes sont sauvegardées en RAM seulement si elles sont modifiées, ce qui permet d'accélérer l'horloge.

\end{enumerate}
