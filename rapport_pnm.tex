\subsection{Structure globale}

Le projet contient cinq exécutables. Dans l'ordre, voici le nom du fichier exécutable, le fichier Haskell qui le produit et sa description :
\begin{itemize}
\item \texttt{simulator}, \texttt{Simulator.hs}, un simulateur de netlists ;
\item \texttt{generator}, \texttt{Cpu.hs}, le programme qui génère la netlist du CPU ;
\item \texttt{optimizer}, \texttt{Optimizer.hs}, un optimisateur de netliists ;
\item \texttt{compiler}, \texttt{Compiler.hs}, le compilateur de netlisits vers C ;
\item \texttt{assembly}, \texttt{Assembly.hs}, le compilateur d'assembleur MIPS vers une ROM.
\end{itemize}

Il y a quelques changements de convention pour la netlist, d'un point de vue syntaxique :
\begin{itemize}
\item \texttt{slice $i$ $j$ $x$} contient les bits de $x$ d'indices $>= i$ et $< j$ ;
\item \texttt{ram} et \texttt{rom} ne prennent plus d'entiers en paramètre en ce qui concerne la taille de la RAM et la taille des mots, ceux-ci sont hardcodés.
\end{itemize}

L'arborescence du projet est la suivante :
\begin{itemize}
\item \texttt{src}
  \begin{itemize}
  \item \texttt{Assembly}, qui contient le code du compilateur MIPS vers un fichier qui représente une ROM qui sera chargé en ROM et exécuté par le processeur
    \begin{itemize}
    \item \texttt{Ast.hs}
    \item \texttt{Compiler.hs}
    \item \texttt{Parser.hs}
    \end{itemize}

  \item \texttt{Assembly.hs}
  \item \texttt{Compiler.hs}
  \item \texttt{Cpu}, le code Haskell qui va générer la netlist du CPU
    \begin{itemize}
    \item \texttt{Adder.hs}
    \item \texttt{Alu.hs}
    \item \texttt{Branch.hs}
    \item \texttt{Control.hs}, multiplexage à partir de l'opcode
    \item \texttt{Instr.hs}
    \item \texttt{Memory.hs}
    \item \texttt{Misc.hs}
    \item \texttt{Mult.hs}, multiplication et division sur plusieurs cycles
    \item \texttt{Nalu.hs}
    \end{itemize}
  \item \texttt{Cpu.hs}
  \item \texttt{Netlist}
    \begin{itemize}
    \item \texttt{Ast.hs}
    \item \texttt{Compiler.hs}, le compilateur depuis la représentation Haskell d'une Netlist vers du C
    \item \texttt{Graph.hs}
    \item \texttt{Jazz.hs}, le code de la monade Jazz qui permet de générer des netlists
    \item \texttt{Opt.hs}, le code de l'optimisateur de netlists
    \item \texttt{Parser.hs}
    \item \texttt{Scheduler.hs}
    \item \texttt{Show.hs}
    \item \texttt{Simulator.hs}
    \item \texttt{Typer.hs}
    \end{itemize}
  \item \texttt{Optimizer.hs}
  \item \texttt{Simulator.hs}
  \end{itemize}
\end{itemize}

\subsection{Compilation}

Lancez la commande \texttt{make}.
Elle génère les différents executables du projet,
notamment \texttt{clock\_exec} et \texttt{fast\_clock\_exec} qui sont les deux horloges.

\subsection{Jazz}

Jazz est notre monade pour générer des netlist, elle remplace MiniJazz.
On implémente deux types importants : Bit et Wire, qui représentent les fils et les nappes de fils.
Ainsi, on peut écrire le type du combinateur qui génère une porte AND :\\
  (/\textbackslash) {\hspace{0.5em}}:{\hspace{-1em}}:{\hspace{0.5em}} Bit $\rightarrow$ Bit $\rightarrow$ Jazz Bit
\newline
On a en fait enrichi ce modèle en créant des type class Bt et Wr associées aux deux types :\\
  (/\textbackslash) {\hspace{0.5em}}:{\hspace{-1em}}:{\hspace{0.5em}} (Bt $a$, Bt $b$) $\implies$ $a$ $\rightarrow$ $b$ $\rightarrow$ Jazz Bit
\newline
Ce qui permet d'écrire des choses telles que :\\
  $x$ $\leftarrow$ $a$ /\textbackslash ($b$ /\textbackslash $c$)

Le module Cpu.Control illustre bien la puissance de Jazz :
on implémente une structure qui contient un champ pour chaque instruction cpu,
à chacune de ces insrtructions on associe un fil et on écrit une fonction opcode\_mux qui
à partir de cette structure construit la netlist d'un multiplexer, qui sélectionnera un fil
à partir de l'instruction décodée.
On peut ainsi spécifier très simplement le comportement du CPU.
Il est alors très facile d'ajouter, de retirer et de modifier des instructions au CPU.

\subsection{CPU}

Nous avons implémenté un sous-ensemble de MIPS.
Une liste des instructions se trouve dans Cpu.Control.
Nous avons notamment implémenté :
\begin{itemize}
\item les opérations binaires "élémentaires" (sauf XOR, qui n'était pas dans le standard) ;
\item l'addition et la soustraction ;
\item les jumps ;
\item des branchements conditionnels ;
\item la division et la multiplication sur plusieurs cycles.
\end{itemize}
Elles ne sont cependant pas encore toutes implémentées, ou tout du moins elles ne vérifient
pas encore toutes la sémantique de MIPS.
Les instructions de lecture/écriture en RAM notamment ont un comportement assez mal défini pour
le moment. De même, la division et la multiplication ne fonctionnent que pour des entiers
non signés.
